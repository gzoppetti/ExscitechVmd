\section{Calling C/C++ based molecule file reader plugins from other programs}\label{callmolfileplugins}


Although the molfile plugins were primarily developed for use within VMD, they can also be used by other scientific software written in C, C++, and Fortran, and potentially from scripting languages.

\subsubsection*{C Bindings}
 The native interface for all of the VMD molfile plugins is implemented in C. Plugins written in C++ also expose their interface routines as C-callable functions so this is transparent to the caller. At present, the best example code illustrating how to use molfile plugins in a standalone program is Cat\-DCD. Cat\-DCD is compiled and linked against the static library version of the plugins, and is built as a standalone  program. Static linkage (ala Cat\-DCD) is by far the easiest method of using VMD plugins in other programs. The C interface to the static library requires inclusion of the libmolfile\_\-plugin.h header file which is automatically generated when the plugin tree is compiled, and the  {\bf molfile\_\-plugin.h} header file found in the top level include directory of the plugin library. The libmolfile\_\-plugin.h file includes  extern declarations for every plugin entry poing provided in the compiled libmolfileplugin.a (libmolfileplugin.lib on WIN32/WIN64)  static link library. These declarations encompass the  'init', 'register', and 'fini' routines for each of the plugins  in the static link library. The libmolfile\_\-plugin.h header and libmolfileplugin.a (or .lib) static library are found in the  compile/[lib\_\-architecture]/molfile directory after successfull compilation. Dynamic plugins are structured similarly, except that all of the dynamic plugins provide their 'init', 'register', and 'fini' routines with the prefix \char`\"{}vmdplugin\_\-\char`\"{} rather than with the unique name of the plugin. This allows these entry points to  be opened and queried dynamically using functions such as dlopen(), dlsym(), and dlclose() on Unix, or by using Load\-Library(), Get\-Proc\-Address(), and Free\-Library() on Windows.

\subsubsection*{Fortran Bindings}
 Axel Kohlmeyer has contributed Fortran bindings for the molfile  plugins which are found in the \char`\"{}f77\char`\"{} subdirectory of the molfile\_\-plugins source directory. Examine the README file in the \char`\"{}f77\char`\"{} directory for complete instructions on compiling and using the Fortran bindings for the molfile plugins in your own programs.

